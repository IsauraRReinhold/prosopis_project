\PassOptionsToPackage{unicode=true}{hyperref} % options for packages loaded elsewhere
\PassOptionsToPackage{hyphens}{url}
%
\documentclass[]{article}
\usepackage{lmodern}
\usepackage{amssymb,amsmath}
\usepackage{ifxetex,ifluatex}
\usepackage{fixltx2e} % provides \textsubscript
\ifnum 0\ifxetex 1\fi\ifluatex 1\fi=0 % if pdftex
  \usepackage[T1]{fontenc}
  \usepackage[utf8]{inputenc}
  \usepackage{textcomp} % provides euro and other symbols
\else % if luatex or xelatex
  \usepackage{unicode-math}
  \defaultfontfeatures{Ligatures=TeX,Scale=MatchLowercase}
\fi
% use upquote if available, for straight quotes in verbatim environments
\IfFileExists{upquote.sty}{\usepackage{upquote}}{}
% use microtype if available
\IfFileExists{microtype.sty}{%
\usepackage[]{microtype}
\UseMicrotypeSet[protrusion]{basicmath} % disable protrusion for tt fonts
}{}
\IfFileExists{parskip.sty}{%
\usepackage{parskip}
}{% else
\setlength{\parindent}{0pt}
\setlength{\parskip}{6pt plus 2pt minus 1pt}
}
\usepackage{hyperref}
\hypersetup{
            pdftitle={11DEGs.r},
            pdfauthor={cris},
            pdfborder={0 0 0},
            breaklinks=true}
\urlstyle{same}  % don't use monospace font for urls
\usepackage[margin=1in]{geometry}
\usepackage{color}
\usepackage{fancyvrb}
\newcommand{\VerbBar}{|}
\newcommand{\VERB}{\Verb[commandchars=\\\{\}]}
\DefineVerbatimEnvironment{Highlighting}{Verbatim}{commandchars=\\\{\}}
% Add ',fontsize=\small' for more characters per line
\usepackage{framed}
\definecolor{shadecolor}{RGB}{248,248,248}
\newenvironment{Shaded}{\begin{snugshade}}{\end{snugshade}}
\newcommand{\AlertTok}[1]{\textcolor[rgb]{0.94,0.16,0.16}{#1}}
\newcommand{\AnnotationTok}[1]{\textcolor[rgb]{0.56,0.35,0.01}{\textbf{\textit{#1}}}}
\newcommand{\AttributeTok}[1]{\textcolor[rgb]{0.77,0.63,0.00}{#1}}
\newcommand{\BaseNTok}[1]{\textcolor[rgb]{0.00,0.00,0.81}{#1}}
\newcommand{\BuiltInTok}[1]{#1}
\newcommand{\CharTok}[1]{\textcolor[rgb]{0.31,0.60,0.02}{#1}}
\newcommand{\CommentTok}[1]{\textcolor[rgb]{0.56,0.35,0.01}{\textit{#1}}}
\newcommand{\CommentVarTok}[1]{\textcolor[rgb]{0.56,0.35,0.01}{\textbf{\textit{#1}}}}
\newcommand{\ConstantTok}[1]{\textcolor[rgb]{0.00,0.00,0.00}{#1}}
\newcommand{\ControlFlowTok}[1]{\textcolor[rgb]{0.13,0.29,0.53}{\textbf{#1}}}
\newcommand{\DataTypeTok}[1]{\textcolor[rgb]{0.13,0.29,0.53}{#1}}
\newcommand{\DecValTok}[1]{\textcolor[rgb]{0.00,0.00,0.81}{#1}}
\newcommand{\DocumentationTok}[1]{\textcolor[rgb]{0.56,0.35,0.01}{\textbf{\textit{#1}}}}
\newcommand{\ErrorTok}[1]{\textcolor[rgb]{0.64,0.00,0.00}{\textbf{#1}}}
\newcommand{\ExtensionTok}[1]{#1}
\newcommand{\FloatTok}[1]{\textcolor[rgb]{0.00,0.00,0.81}{#1}}
\newcommand{\FunctionTok}[1]{\textcolor[rgb]{0.00,0.00,0.00}{#1}}
\newcommand{\ImportTok}[1]{#1}
\newcommand{\InformationTok}[1]{\textcolor[rgb]{0.56,0.35,0.01}{\textbf{\textit{#1}}}}
\newcommand{\KeywordTok}[1]{\textcolor[rgb]{0.13,0.29,0.53}{\textbf{#1}}}
\newcommand{\NormalTok}[1]{#1}
\newcommand{\OperatorTok}[1]{\textcolor[rgb]{0.81,0.36,0.00}{\textbf{#1}}}
\newcommand{\OtherTok}[1]{\textcolor[rgb]{0.56,0.35,0.01}{#1}}
\newcommand{\PreprocessorTok}[1]{\textcolor[rgb]{0.56,0.35,0.01}{\textit{#1}}}
\newcommand{\RegionMarkerTok}[1]{#1}
\newcommand{\SpecialCharTok}[1]{\textcolor[rgb]{0.00,0.00,0.00}{#1}}
\newcommand{\SpecialStringTok}[1]{\textcolor[rgb]{0.31,0.60,0.02}{#1}}
\newcommand{\StringTok}[1]{\textcolor[rgb]{0.31,0.60,0.02}{#1}}
\newcommand{\VariableTok}[1]{\textcolor[rgb]{0.00,0.00,0.00}{#1}}
\newcommand{\VerbatimStringTok}[1]{\textcolor[rgb]{0.31,0.60,0.02}{#1}}
\newcommand{\WarningTok}[1]{\textcolor[rgb]{0.56,0.35,0.01}{\textbf{\textit{#1}}}}
\usepackage{graphicx,grffile}
\makeatletter
\def\maxwidth{\ifdim\Gin@nat@width>\linewidth\linewidth\else\Gin@nat@width\fi}
\def\maxheight{\ifdim\Gin@nat@height>\textheight\textheight\else\Gin@nat@height\fi}
\makeatother
% Scale images if necessary, so that they will not overflow the page
% margins by default, and it is still possible to overwrite the defaults
% using explicit options in \includegraphics[width, height, ...]{}
\setkeys{Gin}{width=\maxwidth,height=\maxheight,keepaspectratio}
\setlength{\emergencystretch}{3em}  % prevent overfull lines
\providecommand{\tightlist}{%
  \setlength{\itemsep}{0pt}\setlength{\parskip}{0pt}}
\setcounter{secnumdepth}{0}
% Redefines (sub)paragraphs to behave more like sections
\ifx\paragraph\undefined\else
\let\oldparagraph\paragraph
\renewcommand{\paragraph}[1]{\oldparagraph{#1}\mbox{}}
\fi
\ifx\subparagraph\undefined\else
\let\oldsubparagraph\subparagraph
\renewcommand{\subparagraph}[1]{\oldsubparagraph{#1}\mbox{}}
\fi

% set default figure placement to htbp
\makeatletter
\def\fps@figure{htbp}
\makeatother


\title{11DEGs.r}
\author{cris}
\date{2022-04-18}

\begin{document}
\maketitle

\begin{Shaded}
\begin{Highlighting}[]
\KeywordTok{library}\NormalTok{(edgeR)}
\end{Highlighting}
\end{Shaded}

\begin{verbatim}
## Loading required package: limma
\end{verbatim}

\begin{Shaded}
\begin{Highlighting}[]
\KeywordTok{library}\NormalTok{(pheatmap)}
\KeywordTok{library}\NormalTok{(ggplot2)}
\KeywordTok{library}\NormalTok{(tidyverse)}
\end{Highlighting}
\end{Shaded}

\begin{verbatim}
## -- Attaching packages --------------------------------------- tidyverse 1.3.1 --
\end{verbatim}

\begin{verbatim}
## v tibble  3.1.6     v dplyr   1.0.8
## v tidyr   1.2.0     v stringr 1.4.0
## v readr   2.1.2     v forcats 0.5.1
## v purrr   0.3.4
\end{verbatim}

\begin{verbatim}
## -- Conflicts ------------------------------------------ tidyverse_conflicts() --
## x dplyr::filter() masks stats::filter()
## x dplyr::lag()    masks stats::lag()
\end{verbatim}

\begin{Shaded}
\begin{Highlighting}[]
\KeywordTok{source}\NormalTok{(}\StringTok{"/home/cris/Documentos/EpiDiso/Disocactus_transcriptome/bin/dif_exp_functions.R"}\NormalTok{)}

\CommentTok{#set workdir}
\KeywordTok{setwd}\NormalTok{(}\StringTok{"~/Documentos/Prosopis_project/bin/"}\NormalTok{)}

\CommentTok{#Charge data}
\NormalTok{count_matrix<-}\KeywordTok{read.table}\NormalTok{(}\StringTok{"../out/count_matrix/prosopis_count_matrix.txt"}\NormalTok{, }\DataTypeTok{header =} \OtherTok{TRUE}\NormalTok{, }
  \DataTypeTok{stringsAsFactors =} \OtherTok{FALSE}\NormalTok{)}
\CommentTok{#load metadata}
\NormalTok{meta <-}\StringTok{ }\KeywordTok{read.table}\NormalTok{(}\StringTok{"../metadata/meta.txt"}\NormalTok{, }\DataTypeTok{header =}\NormalTok{ T)}

\CommentTok{#change colnames}
\KeywordTok{colnames}\NormalTok{(count_matrix) <-}\StringTok{ }\KeywordTok{c}\NormalTok{(}\StringTok{"Gene_id"}\NormalTok{,}\StringTok{"Chr"}\NormalTok{,}\StringTok{"Start"}\NormalTok{,}\StringTok{"End"}\NormalTok{,}\StringTok{"Strand"}\NormalTok{,}\StringTok{"Length"}\NormalTok{,}
                            \StringTok{"Ghaf12DT_002_CGATGT_L007"}\NormalTok{,}\StringTok{"Ghaf2DT_005_ACAGTG_L007"}\NormalTok{,}
                            \StringTok{"Ghaf4DT_006_GCCAAT_L007"}\NormalTok{,}\StringTok{"Ghaf6DT_007_CAGATC_L007"}\NormalTok{,}
                            \StringTok{"Ghaf8DT_009_GATCAG_L007"}\NormalTok{,}\StringTok{"PCDT3.10b"}\NormalTok{,}\StringTok{"PCDT3.12"}\NormalTok{,}\StringTok{"PCDT3.2b"}\NormalTok{,}
                            \StringTok{"PCDT3.4b"}\NormalTok{,}\StringTok{"PCDT3.6"}\NormalTok{,}\StringTok{"PCDT3.8"}\NormalTok{,}\StringTok{"PCDT4.10b"}\NormalTok{,}\StringTok{"PCDT5.4b"}\NormalTok{,}
                            \StringTok{"PCDT5.8b"}\NormalTok{,}\StringTok{"PDT5_10"}\NormalTok{,}\StringTok{"PDT5_2"}\NormalTok{,}\StringTok{"PDT5_6"}\NormalTok{)}

\CommentTok{#convert gen_id in the rownames  }
\KeywordTok{rownames}\NormalTok{(count_matrix)<-count_matrix[,}\DecValTok{1}\NormalTok{]}

\NormalTok{count_matrix<-count_matrix[,}\OperatorTok{-}\DecValTok{1}\NormalTok{]}
\CommentTok{#select only colums whith counts}
\NormalTok{count_matrix<-count_matrix}\OperatorTok
\StringTok{  }\KeywordTok{select}\NormalTok{(., (}\DecValTok{6}\OperatorTok{:}\DecValTok{22}\NormalTok{))}

\KeywordTok{class}\NormalTok{(count_matrix)}
\end{Highlighting}
\end{Shaded}

\begin{verbatim}
## [1] "data.frame"
\end{verbatim}

\begin{Shaded}
\begin{Highlighting}[]
\CommentTok{#explore matrix}
\KeywordTok{names}\NormalTok{(count_matrix)}
\end{Highlighting}
\end{Shaded}

\begin{verbatim}
##  [1] "Ghaf12DT_002_CGATGT_L007" "Ghaf2DT_005_ACAGTG_L007" 
##  [3] "Ghaf4DT_006_GCCAAT_L007"  "Ghaf6DT_007_CAGATC_L007" 
##  [5] "Ghaf8DT_009_GATCAG_L007"  "PCDT3.10b"               
##  [7] "PCDT3.12"                 "PCDT3.2b"                
##  [9] "PCDT3.4b"                 "PCDT3.6"                 
## [11] "PCDT3.8"                  "PCDT4.10b"               
## [13] "PCDT5.4b"                 "PCDT5.8b"                
## [15] "PDT5_10"                  "PDT5_2"                  
## [17] "PDT5_6"
\end{verbatim}

\begin{Shaded}
\begin{Highlighting}[]
\CommentTok{#reoder columns as meta data is }
\NormalTok{col_order <-}\StringTok{ }\KeywordTok{c}\NormalTok{(}\StringTok{"PCDT3.2b"}\NormalTok{,}\StringTok{"PCDT3.4b"}\NormalTok{,}\StringTok{"PCDT3.6"}\NormalTok{, }\StringTok{"PCDT3.8"}\NormalTok{,                     }
               \StringTok{"PCDT3.10b"}\NormalTok{,}\StringTok{"PCDT3.12"}\NormalTok{,}\StringTok{"Ghaf2DT_005_ACAGTG_L007"}\NormalTok{, }\StringTok{"Ghaf4DT_006_GCCAAT_L007"}\NormalTok{ ,     }
               \StringTok{"Ghaf6DT_007_CAGATC_L007"}\NormalTok{,}\StringTok{"Ghaf8DT_009_GATCAG_L007"}\NormalTok{,}\StringTok{"Ghaf12DT_002_CGATGT_L007"}\NormalTok{,     }
               \StringTok{"PCDT4.10b"}\NormalTok{,}\StringTok{"PCDT5.4b"}\NormalTok{,}\StringTok{"PCDT5.8b"}\NormalTok{,}\StringTok{"PDT5_10"}\NormalTok{, }\StringTok{"PDT5_2"}\NormalTok{,}\StringTok{"PDT5_6"}\NormalTok{)}

\NormalTok{count_matrix<-}\StringTok{ }\NormalTok{count_matrix[, col_order]}

\CommentTok{#filter data by cpm}
\NormalTok{keep <-}\StringTok{ }\KeywordTok{rowSums}\NormalTok{(}\KeywordTok{cpm}\NormalTok{(count_matrix) }\OperatorTok{>=}\StringTok{ }\DecValTok{5}\NormalTok{) }\OperatorTok{>=}\DecValTok{2}
\KeywordTok{table}\NormalTok{(keep)}
\end{Highlighting}
\end{Shaded}

\begin{verbatim}
## keep
## FALSE  TRUE 
## 54864 22354
\end{verbatim}

\begin{Shaded}
\begin{Highlighting}[]
\NormalTok{count_matrix <-}\StringTok{ }\NormalTok{count_matrix[keep, ]}

\CommentTok{#explore count matrix}
\KeywordTok{colnames}\NormalTok{(count_matrix)}
\end{Highlighting}
\end{Shaded}

\begin{verbatim}
##  [1] "PCDT3.2b"                 "PCDT3.4b"                
##  [3] "PCDT3.6"                  "PCDT3.8"                 
##  [5] "PCDT3.10b"                "PCDT3.12"                
##  [7] "Ghaf2DT_005_ACAGTG_L007"  "Ghaf4DT_006_GCCAAT_L007" 
##  [9] "Ghaf6DT_007_CAGATC_L007"  "Ghaf8DT_009_GATCAG_L007" 
## [11] "Ghaf12DT_002_CGATGT_L007" "PCDT4.10b"               
## [13] "PCDT5.4b"                 "PCDT5.8b"                
## [15] "PDT5_10"                  "PDT5_2"                  
## [17] "PDT5_6"
\end{verbatim}

\begin{Shaded}
\begin{Highlighting}[]
\NormalTok{groups <-}\StringTok{ }\KeywordTok{factor}\NormalTok{(}\KeywordTok{colnames}\NormalTok{(count_matrix))}
\KeywordTok{table}\NormalTok{(groups)}
\end{Highlighting}
\end{Shaded}

\begin{verbatim}
## groups
## Ghaf12DT_002_CGATGT_L007  Ghaf2DT_005_ACAGTG_L007  Ghaf4DT_006_GCCAAT_L007 
##                        1                        1                        1 
##  Ghaf6DT_007_CAGATC_L007  Ghaf8DT_009_GATCAG_L007                PCDT3.10b 
##                        1                        1                        1 
##                 PCDT3.12                 PCDT3.2b                 PCDT3.4b 
##                        1                        1                        1 
##                  PCDT3.6                  PCDT3.8                PCDT4.10b 
##                        1                        1                        1 
##                 PCDT5.4b                 PCDT5.8b                  PDT5_10 
##                        1                        1                        1 
##                   PDT5_2                   PDT5_6 
##                        1                        1
\end{verbatim}

\begin{Shaded}
\begin{Highlighting}[]
\CommentTok{#create edgeR list}
\NormalTok{edgeRlist <-}\StringTok{ }\KeywordTok{DGEList}\NormalTok{(}\DataTypeTok{counts =}\NormalTok{ count_matrix,}
                     \DataTypeTok{group =}\NormalTok{ meta}\OperatorTok{$}\NormalTok{month.Treatment, }
                     \DataTypeTok{genes =} \KeywordTok{rownames}\NormalTok{(count_matrix))}
\KeywordTok{str}\NormalTok{(edgeRlist)}
\end{Highlighting}
\end{Shaded}

\begin{verbatim}
## Formal class 'DGEList' [package "edgeR"] with 1 slot
##   ..@ .Data:List of 3
##   .. ..$ : int [1:22354, 1:17] 229 1920 2693 1275 1520 79 1096 755 154 143 ...
##   .. .. ..- attr(*, "dimnames")=List of 2
##   .. .. .. ..$ : chr [1:22354] "KCPC_00000005-RA" "KCPC_00000006-RA" "KCPC_00000013-RA" "KCPC_00000015-RA" ...
##   .. .. .. ..$ : chr [1:17] "PCDT3.2b" "PCDT3.4b" "PCDT3.6" "PCDT3.8" ...
##   .. ..$ :'data.frame':  17 obs. of  3 variables:
##   .. .. ..$ group       : Factor w/ 6 levels "month_10","month_12",..: 3 4 5 6 1 2 3 4 5 6 ...
##   .. .. ..$ lib.size    : num [1:17] 41879743 43124317 40559418 32842486 34102995 ...
##   .. .. ..$ norm.factors: num [1:17] 1 1 1 1 1 1 1 1 1 1 ...
##   .. ..$ :'data.frame':  22354 obs. of  1 variable:
##   .. .. ..$ genes: chr [1:22354] "KCPC_00000005-RA" "KCPC_00000006-RA" "KCPC_00000013-RA" "KCPC_00000015-RA" ...
##   ..$ names: chr [1:3] "counts" "samples" "genes"
\end{verbatim}

\begin{Shaded}
\begin{Highlighting}[]
\CommentTok{#Normalized count by TMM}
\NormalTok{edgeRlist <-}\StringTok{ }\KeywordTok{calcNormFactors}\NormalTok{(edgeRlist, }\DataTypeTok{method =} \StringTok{"TMM"}\NormalTok{)}
\NormalTok{edgeRlist}\OperatorTok{$}\NormalTok{samples}
\end{Highlighting}
\end{Shaded}

\begin{verbatim}
##                             group lib.size norm.factors
## PCDT3.2b                  month_2 41879743    0.8944485
## PCDT3.4b                  month_4 43124317    0.8330293
## PCDT3.6                   month_6 40559418    1.2431947
## PCDT3.8                   month_8 32842486    0.7856211
## PCDT3.10b                month_10 34102995    0.7384695
## PCDT3.12                 month_12 34887090    0.8875539
## Ghaf2DT_005_ACAGTG_L007   month_2 30373184    1.2821889
## Ghaf4DT_006_GCCAAT_L007   month_4 25640463    1.2275675
## Ghaf6DT_007_CAGATC_L007   month_6 31010380    1.3678322
## Ghaf8DT_009_GATCAG_L007   month_8 27172366    1.2617610
## Ghaf12DT_002_CGATGT_L007 month_10 29775513    0.5684423
## PCDT4.10b                month_12 38695465    1.0147384
## PCDT5.4b                  month_2 44846973    1.0151217
## PCDT5.8b                  month_4 37983556    1.1336470
## PDT5_10                   month_6 35384768    1.0500525
## PDT5_2                    month_8 33239856    0.8629483
## PDT5_6                   month_10 40184553    1.2831142
\end{verbatim}

\begin{Shaded}
\begin{Highlighting}[]
\CommentTok{## Plot to evaluate the correct data normalization }
\CommentTok{## Plot the results using absolute vs relative expression in each sample to check the correct normalization}

\KeywordTok{pdf}\NormalTok{(}\StringTok{"../figures/MD_plots.pdf"}\NormalTok{, }\DataTypeTok{height =} \DecValTok{7}\NormalTok{, }\DataTypeTok{width =} \DecValTok{10}\NormalTok{)}
\KeywordTok{par}\NormalTok{(}\DataTypeTok{mfrow =} \KeywordTok{c}\NormalTok{(}\DecValTok{2}\NormalTok{, }\DecValTok{3}\NormalTok{)) }\CommentTok{##Generate a frame to store 6 plots in 2 rows and 3 columns}
\ControlFlowTok{for}\NormalTok{ (i }\ControlFlowTok{in} \KeywordTok{c}\NormalTok{(}\DecValTok{1}\OperatorTok{:}\DecValTok{17}\NormalTok{)) \{}
  \KeywordTok{print}\NormalTok{(}\KeywordTok{plotMD}\NormalTok{(}\KeywordTok{cpm}\NormalTok{(edgeRlist, }\DataTypeTok{log =}\NormalTok{ T), }\DataTypeTok{column =}\NormalTok{ i))}
  \KeywordTok{grid}\NormalTok{(}\DataTypeTok{col =} \StringTok{"blue"}\NormalTok{)}
  \KeywordTok{abline}\NormalTok{(}\DataTypeTok{h =} \DecValTok{0}\NormalTok{, }\DataTypeTok{col =} \StringTok{"red"}\NormalTok{, }\DataTypeTok{lty =} \DecValTok{2}\NormalTok{, }\DataTypeTok{lwd =} \DecValTok{2}\NormalTok{)}
\NormalTok{\}}
\end{Highlighting}
\end{Shaded}

\begin{verbatim}
## NULL
\end{verbatim}

\begin{verbatim}
## NULL
\end{verbatim}

\begin{verbatim}
## NULL
\end{verbatim}

\begin{verbatim}
## NULL
\end{verbatim}

\begin{verbatim}
## NULL
\end{verbatim}

\begin{verbatim}
## NULL
\end{verbatim}

\begin{verbatim}
## NULL
\end{verbatim}

\begin{verbatim}
## NULL
\end{verbatim}

\begin{verbatim}
## NULL
\end{verbatim}

\begin{verbatim}
## NULL
\end{verbatim}

\begin{verbatim}
## NULL
\end{verbatim}

\begin{verbatim}
## NULL
\end{verbatim}

\begin{verbatim}
## NULL
\end{verbatim}

\begin{verbatim}
## NULL
\end{verbatim}

\begin{verbatim}
## NULL
\end{verbatim}

\begin{verbatim}
## NULL
\end{verbatim}

\begin{verbatim}
## NULL
\end{verbatim}

\begin{Shaded}
\begin{Highlighting}[]
\KeywordTok{dev.off}\NormalTok{()}
\end{Highlighting}
\end{Shaded}

\begin{verbatim}
## pdf 
##   2
\end{verbatim}

\begin{Shaded}
\begin{Highlighting}[]
\CommentTok{#Heatmap to explor data}
\CommentTok{#calculating the correlation (Pearson) that exists between the samples}
\KeywordTok{pdf}\NormalTok{(}\StringTok{"../figures/corr_rep_plots.pdf"}\NormalTok{, }\DataTypeTok{height =} \DecValTok{7}\NormalTok{, }\DataTypeTok{width =} \DecValTok{10}\NormalTok{)}
\NormalTok{cormat <-}\StringTok{ }\KeywordTok{cor}\NormalTok{(}\KeywordTok{cpm}\NormalTok{(edgeRlist}\OperatorTok{$}\NormalTok{counts, }\DataTypeTok{log =}\NormalTok{ T))}
\KeywordTok{pheatmap}\NormalTok{(cormat, }\DataTypeTok{border_color =} \OtherTok{NA}\NormalTok{, }\DataTypeTok{main =} \StringTok{"P. cineraria correlation of replicates"}\NormalTok{)}
\KeywordTok{dev.off}\NormalTok{()}
\end{Highlighting}
\end{Shaded}

\begin{verbatim}
## pdf 
##   3
\end{verbatim}

\begin{Shaded}
\begin{Highlighting}[]
\CommentTok{#In order to check the correct normalization of the samples we repeat the boxplot}
\CommentTok{# Get log2 counts per million}
\KeywordTok{jpeg}\NormalTok{(}\StringTok{"../figures/boxplot_logCPMs_norm.jpg"}\NormalTok{)}
\NormalTok{logcounts <-}\StringTok{ }\KeywordTok{cpm}\NormalTok{(edgeRlist,}\DataTypeTok{log=}\OtherTok{TRUE}\NormalTok{)}
\CommentTok{# Check distributions of samples using boxplots}
\KeywordTok{boxplot}\NormalTok{(logcounts, }\DataTypeTok{xlab=}\StringTok{""}\NormalTok{, }\DataTypeTok{ylab=}\StringTok{"Log2 counts per million"}\NormalTok{,}\DataTypeTok{las=}\DecValTok{2}\NormalTok{)}
\CommentTok{# Let's add a blue horizontal line that corresponds to the median logCPM}
\KeywordTok{abline}\NormalTok{(}\DataTypeTok{h=}\KeywordTok{median}\NormalTok{(logcounts),}\DataTypeTok{col=}\StringTok{"blue"}\NormalTok{)}
\KeywordTok{title}\NormalTok{(}\StringTok{"transformed logCPMs"}\NormalTok{)}
\KeywordTok{dev.off}\NormalTok{()}
\end{Highlighting}
\end{Shaded}

\begin{verbatim}
## pdf 
##   3
\end{verbatim}

\begin{Shaded}
\begin{Highlighting}[]
\CommentTok{#Experiimental matrix design}
\NormalTok{design <-}\StringTok{ }\KeywordTok{model.matrix}\NormalTok{(}\OperatorTok{~}\DecValTok{0}\OperatorTok{+}\NormalTok{edgeRlist}\OperatorTok{$}\NormalTok{samples}\OperatorTok{$}\NormalTok{group)}
\NormalTok{design}
\end{Highlighting}
\end{Shaded}

\begin{verbatim}
##    edgeRlist$samples$groupmonth_10 edgeRlist$samples$groupmonth_12
## 1                                0                               0
## 2                                0                               0
## 3                                0                               0
## 4                                0                               0
## 5                                1                               0
## 6                                0                               1
## 7                                0                               0
## 8                                0                               0
## 9                                0                               0
## 10                               0                               0
## 11                               1                               0
## 12                               0                               1
## 13                               0                               0
## 14                               0                               0
## 15                               0                               0
## 16                               0                               0
## 17                               1                               0
##    edgeRlist$samples$groupmonth_2 edgeRlist$samples$groupmonth_4
## 1                               1                              0
## 2                               0                              1
## 3                               0                              0
## 4                               0                              0
## 5                               0                              0
## 6                               0                              0
## 7                               1                              0
## 8                               0                              1
## 9                               0                              0
## 10                              0                              0
## 11                              0                              0
## 12                              0                              0
## 13                              1                              0
## 14                              0                              1
## 15                              0                              0
## 16                              0                              0
## 17                              0                              0
##    edgeRlist$samples$groupmonth_6 edgeRlist$samples$groupmonth_8
## 1                               0                              0
## 2                               0                              0
## 3                               1                              0
## 4                               0                              1
## 5                               0                              0
## 6                               0                              0
## 7                               0                              0
## 8                               0                              0
## 9                               1                              0
## 10                              0                              1
## 11                              0                              0
## 12                              0                              0
## 13                              0                              0
## 14                              0                              0
## 15                              1                              0
## 16                              0                              1
## 17                              0                              0
## attr(,"assign")
## [1] 1 1 1 1 1 1
## attr(,"contrasts")
## attr(,"contrasts")$`edgeRlist$samples$group`
## [1] "contr.treatment"
\end{verbatim}

\begin{Shaded}
\begin{Highlighting}[]
\CommentTok{##the term ~0 tells the function not to include a column of intersections and only include as many columns as groups in our experimental design}
\KeywordTok{colnames}\NormalTok{(design) <-}\StringTok{ }\KeywordTok{levels}\NormalTok{(edgeRlist}\OperatorTok{$}\NormalTok{samples}\OperatorTok{$}\NormalTok{group)}

\CommentTok{#explore design}
\NormalTok{design}
\end{Highlighting}
\end{Shaded}

\begin{verbatim}
##    month_10 month_12 month_2 month_4 month_6 month_8
## 1         0        0       1       0       0       0
## 2         0        0       0       1       0       0
## 3         0        0       0       0       1       0
## 4         0        0       0       0       0       1
## 5         1        0       0       0       0       0
## 6         0        1       0       0       0       0
## 7         0        0       1       0       0       0
## 8         0        0       0       1       0       0
## 9         0        0       0       0       1       0
## 10        0        0       0       0       0       1
## 11        1        0       0       0       0       0
## 12        0        1       0       0       0       0
## 13        0        0       1       0       0       0
## 14        0        0       0       1       0       0
## 15        0        0       0       0       1       0
## 16        0        0       0       0       0       1
## 17        1        0       0       0       0       0
## attr(,"assign")
## [1] 1 1 1 1 1 1
## attr(,"contrasts")
## attr(,"contrasts")$`edgeRlist$samples$group`
## [1] "contr.treatment"
\end{verbatim}

\begin{Shaded}
\begin{Highlighting}[]
\CommentTok{#calculate data dispesion}
\KeywordTok{jpeg}\NormalTok{(}\StringTok{"../figures/data_dispersion.jpg"}\NormalTok{)}
\NormalTok{edgeRlist <-}\StringTok{ }\KeywordTok{estimateDisp}\NormalTok{(edgeRlist, }\DataTypeTok{design =}\NormalTok{ design, }\DataTypeTok{robust =}\NormalTok{ T)}
\KeywordTok{plotBCV}\NormalTok{(edgeRlist)}
\KeywordTok{dev.off}\NormalTok{()}
\end{Highlighting}
\end{Shaded}

\begin{verbatim}
## pdf 
##   3
\end{verbatim}

\begin{Shaded}
\begin{Highlighting}[]
\CommentTok{#estimation of QL dispersions}
\CommentTok{#Data must fit a negative bi-nominal linear model. For this, the glmQLfit function will be used with which there is a more robust control of the error}
\NormalTok{fit <-}\StringTok{ }\KeywordTok{glmQLFit}\NormalTok{(edgeRlist, design, }\DataTypeTok{robust=}\OtherTok{TRUE}\NormalTok{)}
\KeywordTok{head}\NormalTok{(fit}\OperatorTok{$}\NormalTok{coefficients)}
\end{Highlighting}
\end{Shaded}

\begin{verbatim}
##                    month_10   month_12    month_2    month_4    month_6
## KCPC_00000005-RA -12.230858 -12.049787 -12.347226 -12.246639 -12.418340
## KCPC_00000006-RA  -9.992375  -9.838976 -10.000756  -9.977719 -10.154735
## KCPC_00000013-RA  -9.790987  -9.651588  -9.376311  -9.215017  -9.431225
## KCPC_00000015-RA  -9.649078  -9.953982  -9.973272 -10.068398  -9.965221
## KCPC_00000019-RA -10.368387 -10.275671  -9.959031 -10.409000 -10.292389
## KCPC_00000020-RA -11.460259 -12.564344 -12.029589 -11.988838 -11.526421
##                     month_8
## KCPC_00000005-RA -12.313240
## KCPC_00000006-RA -10.006816
## KCPC_00000013-RA  -9.679837
## KCPC_00000015-RA  -9.804698
## KCPC_00000019-RA -10.282494
## KCPC_00000020-RA -11.507426
\end{verbatim}

\begin{Shaded}
\begin{Highlighting}[]
\CommentTok{#Plot QL dispersion  using fit object}
\KeywordTok{jpeg}\NormalTok{(}\StringTok{"../figures/QL_disp.jpg"}\NormalTok{)}
\KeywordTok{plotQLDisp}\NormalTok{(fit, }\DataTypeTok{main =} \StringTok{" Quasi Likelihood dispersion in P.cinerase"}\NormalTok{)}
\KeywordTok{dev.off}\NormalTok{()}
\end{Highlighting}
\end{Shaded}

\begin{verbatim}
## pdf 
##   3
\end{verbatim}

\begin{Shaded}
\begin{Highlighting}[]
\CommentTok{#Contrast matrix for 3 comparisons }
\CommentTok{#Since we are interested in differences between groups, we need to specify which comparisons we want to test.}
\NormalTok{contrast_matrix <-}\StringTok{ }\KeywordTok{makeContrasts}\NormalTok{(}
  \StringTok{"PC_2vsPC_4"}\NormalTok{ =}\StringTok{ "month_2 - month_4"}\NormalTok{,}
  \StringTok{"PC_4vsPC_6"}\NormalTok{ =}\StringTok{ "month_4 - month_6"}\NormalTok{, }
  \StringTok{"PC_6vsPC_8"}\NormalTok{ =}\StringTok{ "month_6 - month_8"}\NormalTok{,}
  \StringTok{"PC_8vsPC_10"}\NormalTok{ =}\StringTok{ "month_8 - month_10"}\NormalTok{,}\DataTypeTok{levels=}\NormalTok{design)}


\NormalTok{contrast_matrix}
\end{Highlighting}
\end{Shaded}

\begin{verbatim}
##           Contrasts
## Levels     PC_2vsPC_4 PC_4vsPC_6 PC_6vsPC_8 PC_8vsPC_10
##   month_10          0          0          0          -1
##   month_12          0          0          0           0
##   month_2           1          0          0           0
##   month_4          -1          1          0           0
##   month_6           0         -1          1           0
##   month_8           0          0         -1           1
\end{verbatim}

\begin{Shaded}
\begin{Highlighting}[]
\CommentTok{#Create the object contr_leves}
\NormalTok{contr_levels<-}\KeywordTok{attributes}\NormalTok{(contrast_matrix)}\OperatorTok{$}\NormalTok{dimnames}\OperatorTok{$}\NormalTok{Contrasts}

\CommentTok{#Adjust data to Binomial (BN) method and generate volcano plots for every comparisson}
\KeywordTok{pdf}\NormalTok{(}\StringTok{"../figures/prosopis_volcanos.pdf"}\NormalTok{, }\DataTypeTok{height =} \DecValTok{7}\NormalTok{, }\DataTypeTok{width =} \DecValTok{10}\NormalTok{)}
\KeywordTok{par}\NormalTok{(}\DataTypeTok{mfrow =} \KeywordTok{c}\NormalTok{(}\DecValTok{2}\NormalTok{, }\DecValTok{3}\NormalTok{)) }\CommentTok{##Generate a frame to store 3 plots in 2 rows and 3 columns}
\NormalTok{dif_exp_results<-}\KeywordTok{data.frame}\NormalTok{() }\CommentTok{#Empty data frame}
\ControlFlowTok{for}\NormalTok{ (i }\ControlFlowTok{in} \KeywordTok{c}\NormalTok{(}\DecValTok{1}\OperatorTok{:}\DecValTok{4}\NormalTok{)) \{}
\NormalTok{  qlf.BvsA.lfc1 <-}\StringTok{ }\KeywordTok{glmTreat}\NormalTok{(fit, }\CommentTok{##Object in list form with data fitted to a negative bi-nominal model}
                            \DataTypeTok{contrast =}\NormalTok{ contrast_matrix[, i], }
                            \DataTypeTok{lfc =} \DecValTok{1}\NormalTok{)}
  \CommentTok{##We obtain the SDRs with an expression other than 1, p value less than 0.05, correcting the p value by the Benjamini-Hochberg method}
\NormalTok{  deg.BvsA.lfc1 <-}\StringTok{ }\KeywordTok{decideTestsDGE}\NormalTok{(qlf.BvsA.lfc1, }\DataTypeTok{p.value =} \FloatTok{0.05}\NormalTok{, }\DataTypeTok{adjust.method =} \StringTok{"BH"}\NormalTok{, }\DataTypeTok{lfc =} \DecValTok{1}\NormalTok{)}
  \KeywordTok{table}\NormalTok{(deg.BvsA.lfc1)}
  \CommentTok{#select the genes that statistically have | lfc | > 1 and strengthen our results}
\NormalTok{  DEG.BvsA.lfc1 <-}\StringTok{ }\KeywordTok{DEGResults}\NormalTok{(qlf.BvsA.lfc1)}
\NormalTok{  DEG.BvsA.lfc1 <-}\StringTok{ }\KeywordTok{edgeResults}\NormalTok{(DEG.BvsA.lfc1, }\DataTypeTok{logfc =} \DecValTok{1}\NormalTok{, }\DataTypeTok{padj =} \FloatTok{0.05}\NormalTok{)}
\NormalTok{  comparacion<-}\KeywordTok{data.frame}\NormalTok{(}\DataTypeTok{comparacion =} \KeywordTok{rep}\NormalTok{(contr_levels[i], }\DataTypeTok{times =} \KeywordTok{nrow}\NormalTok{(DEG.BvsA.lfc1)))}
\NormalTok{  DEG.BvsA.lfc1 <-}\KeywordTok{cbind}\NormalTok{(comparacion, DEG.BvsA.lfc1)}
\NormalTok{  dif_exp_results<-}\KeywordTok{rbind}\NormalTok{(dif_exp_results, DEG.BvsA.lfc1)}
  \KeywordTok{print}\NormalTok{(}\KeywordTok{volcano_edgeR}\NormalTok{(DEG.BvsA.lfc1, }\DataTypeTok{lfc =} \DecValTok{1}\NormalTok{, }\DataTypeTok{padj =} \FloatTok{0.05}\NormalTok{) }\OperatorTok{+}\StringTok{ }\CommentTok{#print volcano_plots}
\StringTok{          }\KeywordTok{labs}\NormalTok{(}\DataTypeTok{title =}\NormalTok{ contr_levels[i]) }\OperatorTok{+}\StringTok{ }\CommentTok{#labs of comparissons}
\StringTok{          }\KeywordTok{xlim}\NormalTok{(}\OperatorTok{-}\DecValTok{15}\NormalTok{, }\DecValTok{15}\NormalTok{) }\OperatorTok{+}
\StringTok{          }\KeywordTok{ylim}\NormalTok{(}\DecValTok{0}\NormalTok{, }\DecValTok{10}\NormalTok{))}
  
\NormalTok{\}}
\KeywordTok{dev.off}\NormalTok{()}
\end{Highlighting}
\end{Shaded}

\begin{verbatim}
## pdf 
##   3
\end{verbatim}

\end{document}
